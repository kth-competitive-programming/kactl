% This is the LaTeX file for the chapter "Useful mathematical identities" in KACTL (KTH ACM Contest Template Library)
% Written by Anders Sjoqvist and Ulf Lundstrom, 2009

\chapter{Useful mathematical identities}

\section{Equations}
$$\begin{aligned}ax+by=c\\cx+dy=f\end{aligned}
\Rightarrow
\begin{aligned}x=\dfrac{ed-bf}{ad-bc}\\y=\dfrac{af-ec}{ad-bc}\end{aligned}$$

\section{Trigonometry}
\begin{align*}
\sin(v+w)&{}=\sin v\cos w+\cos v\sin w\\
cos(v+w)&{}=\cos v\cos w-\sin v\sin w\\
\tan(v+w)&{}=\dfrac{\tan v+\tan w}{1-\tan v\tan w}\\
\sin v+\sin w&{}=2\sin\dfrac{v+w}{2}\cos{v-w}{2}\\
\cos v+\cos w&{}=2\cos\dfrac{v+w}{2}\cos{v-w}{2}\\
(V+W)\tan(v-w)/2&{}=(V-W)\tan(v+w)/2
\end{align*}
where $V$ is the length of the side opposite the angle $v$, and
similar for $W$.

\begin{align*}
\begin{cases}
a\cos x+b\sin x=r\cos(x-\phi)\\
a\sin x+b\cos x=r\sin(x+\phi)
\end{cases}
\end{align*}
where $r=\sqrt{a^2+b^2}, \phi=\operatorname{atan2}(b,a)$.

\begin{center}
\begin{tabular}{|l|l|l|l|}
\hline
$\phi$ & $\sin\phi$ & $\cos\phi$ & $\tan\phi$ \\
\hline
$0 = 0^\circ$ & $0$ & $1$ & $0$\\
\hline
$\pi/6 = 30^\circ$ & $1/2$ & $\sqrt3/2$ & $1/sqrt3$\\
\hline
$\pi/4 = 45^\circ$ & $1/\sqrt2$ & $1/\sqrt2$ & $1$\\
\hline
$\pi/3 = 60^\circ$ & $\sqrt3/2$ & $1/2$ & $\sqrt3$\\
\hline
$\pi/2 = 90^\circ$ & $1$ & $0$ & --\\
\hline
\end{tabular}
\end{center}

\section{Spherical trigonometry}
$a,b,c=\text{sides}$, $\alpha,\beta,\gamma=\text{angles}$, all
six\ldots \emph{NOT DONE YET}
\begin{align*}
\cos a&{}=\cos b \cos c + \sin b \sin c \cos \alpha\\
\cos \alpha&{}=-\cos\beta \cos\gamma + \sin\beta\sin\gamma\cos a\\
\sin \alpha/\sin a&{}=\sin\beta\sin b=\sin\gamma/\sin c
\end{align*}

\section{Geometry}

\subsection{Triangles}
Side lengths: $a,b,c$\\
Semiperimeter: $p=\dfrac{a+b+c}{2}$\\
Area: $A=\sqrt{p(p-a)(p-b)(p-c)}$\\
Circumradius: $R=\dfrac{abc}{4A}$\\
Inradius: $r=\dfrac{A}{p}$\\
Median (divides triangle into two equal-sized triangles): $m_a=\tfrac{1}{2}\sqrt{2b^2+2c^2-a^2}$\\
Bisector (divides angles in two): $s_a=\sqrt{bc\left[1-\left(\dfrac{a}{b+c}\right)^2\right]}$\\
law of sines: $\dfrac{\sin\alpha}{a}=\dfrac{\sin\beta}{b}=\dfrac{\sin\gamma}{c}=\dfrac{1}{2R}$\\
Law of cosines: $a^2=b^2+c^2-2bc\cos\alpha$\\
Law of tangents: $\dfrac{a+b}{a-b}=\dfrac{\tan\dfrac{\alpha+\beta}{2}}{\tan\dfrac{\alpha-\beta}{2}}$\\

\subsection{Quadrilaterals}
Side lengths $a,b,c,d$.\\
Diagonals $e(ad\leftrightarrow bc), f(ab\leftrightarrow cd)$.\\
Diagonals angle $\theta$.\\
Magic flux $F=b^2+d^2-a^2-c^2$.\\
Area $4A=2ef\sin\theta=F\tan\theta=\sqrt{4e^2f^2-F^2}$.

\section{Derivatives/Integrals}
\begin{align*}
\dfrac{d}{dx}\arcsin x = \dfrac{1}{\sqrt{1-x^2}} & \quad
\dfrac{d}{dx}\arccos x = -\dfrac{1}{\sqrt{1-x^2}}\\
\dfrac{d}{dx}\tan x = 1+\tan^2 x & \quad
\dfrac{d}{dx}\arctan x = \dfrac{1}{1+x^2}\\
\int\tan ax = -\dfrac{\ln|\cos ax|}{a} & \quad
\int x\sin ax = \dfrac{\sin ax-ax \cos ax}{a^2}\\
\int \frac{dx}{x} = \ln|x| & \quad
\int xe^{ax}dx = \frac{e^{ax}}{a^2}(ax-1)\\
\end{align*}

Integration by parts:
$$\int_a^bf(x)g(x)dx = [F(x)g(x)]_a^b-\int_a^bF(x)g'(x)dx$$

\section{Series} 
Geometric series:
\begin{align*}
\sum_{k=0}^{n-1}x^k &= \frac{x^n-1}{x-1} = \frac{1-x^n}{1-x},\,(x\neq0)\\
\sum_{k=0}^{\infty}x^k &= \frac{1}{1-x},\,(-1<x<1)\\
\end{align*}

Sums of powers:
\begin{align*}
\sum_{k=1}^n k &= \frac{n(n+1)}{2}\\
\sum_{k=1}^n k^2 &= \frac{n(n+1)(2n+1)}{6}\\
\sum_{k=1}^n k^3 &= \frac{n^2(n+1)^2}{4}\\
\end{align*}

Taylor series:

\section{Ceil and floor relations}


